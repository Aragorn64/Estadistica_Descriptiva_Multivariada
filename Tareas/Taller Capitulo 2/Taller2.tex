\documentclass[letterpaper, 12pt]{article}
\usepackage[utf8]{inputenc}
\usepackage{comment}
\usepackage{amsmath}  % Matrices
\usepackage{amsfonts} % Matrices
\usepackage{booktabs} % Tablas
\usepackage{paralist} % Enumeracion Horizontal
                                                                                
\title{Caracterización de la función de razas de perros}
\author{Esteban Avendaño Forero, Juan David Sarmiento\\
Sebastian Aristizabal.}
\date{\today}

                                
\linespread{1.5}
\usepackage{Sweave}
\begin{document}
\Sconcordance{concordance:Taller2.tex:Taller2.Rnw:%
1 14 1 1 4 1 0 4 1 1 3 2 0 4 1 12 0 2 2 45 0 1 2 %
26 1 1 3 5 0 1 2 1 1 1 2 8 0 1 2 9 1 1 2 5 0 1 2 %
2 1}


\maketitle

\begin{Schunk}
\begin{Sinput}
> #install.packages("FactoClass")
> library(FactoClass)
> data(DogBreeds)
> a <- DogBreeds[,-7]
> b <- DogBreeds[,7]
> chisq.carac(df=a, cl=b, thr=0)
\end{Sinput}
\begin{Soutput}
          chi2 dfr         pval      tval      phi2
WEIG 24.407143   4 6.618319e-05 3.8220136 0.9039683
AFFE 14.760989   2 6.232926e-04 3.2280010 0.5467033
SIZE 16.354286   4 2.578811e-03 2.7970208 0.6057143
AGGR  7.072665   2 2.911993e-02 1.8938881 0.2619505
SPEE  8.483333   4 7.539403e-02 1.4367534 0.3141975
INTE  4.140385   4 3.873401e-01 0.2862584 0.1533476
\end{Soutput}
\end{Schunk}
\newpage 
\begin{Schunk}
\begin{Sinput}
> cluster.carac(a, b, "ca", 2.0)
\end{Sinput}
\begin{Soutput}
class: com
         Test.Value p.Value Class.Cat Cat.Class Global
AFFE.hig      3.849   0.000      71.4       100   51.9
WEIG.lig      3.308   0.001      87.5        70   29.6
SIZE.sma      2.866   0.004      85.7        60   25.9
SIZE.lar     -3.541   0.000       6.7        10   55.6
AFFE.low     -3.849   0.000       0.0         0   48.1
         Weight
AFFE.hig     14
WEIG.lig      8
SIZE.sma      7
SIZE.lar     15
AFFE.low     13
----------------------------------------- 
class: hun
         Test.Value p.Value Class.Cat Cat.Class Global
WEIG.med      2.621   0.009      57.1      88.9   51.9
AFFE.low      2.082   0.037      53.8      77.8   48.1
AFFE.hig     -2.082   0.037      14.3      22.2   51.9
         Weight
WEIG.med     14
AFFE.low     13
AFFE.hig     14
----------------------------------------- 
class: uti
         Test.Value p.Value Class.Cat Cat.Class Global
WEIG.hea      3.392   0.001     100.0      62.5   18.5
SIZE.lar      2.978   0.003      53.3     100.0   55.6
AGGR.hig      2.530   0.011      53.8      87.5   48.1
WEIG.lig     -2.120   0.034       0.0       0.0   29.6
SPEE.med     -2.120   0.034       0.0       0.0   29.6
AGGR.low     -2.530   0.011       7.1      12.5   51.9
         Weight
WEIG.hea      5
SIZE.lar     15
AGGR.hig     13
WEIG.lig      8
SPEE.med      8
AGGR.low     14
\end{Soutput}
\end{Schunk}

\newpage 
\begin{enumerate}

\item Las tres variables que más caracterizan a la función para la cual se
utilizan las razas son: 
\begin{inparaenum}[1)]
\item El peso, 
\item la afectividad y 
\item el tamaño.
\end{inparaenum}

\item La estadística $\chi$ asociada a la tabla de contingencia peso × funcion
es:
\item Para encontrar el valor p se utiliza la distribución: con grados de libertad.
\item El valor test se puede obtener con el comando de R:
Conteste falso o verdadero a las afirmaciones siguientes:
\item De las razas utilizadas para compañía el 71.4 \% es de afectividad alta.\\
\textbf{FALSO} - El 71.4 \% de las razas de afectividad alta son utilizadas para 
compañia.
\item Todas las razas utilizadas para compañía son de afectividad alta.\\
\textbf{VERDADERO} - De las tablas, se evidencia que Cat/Class=100.
\item Todas las razas de afectiividad alta se utilizan para compañía.\\
\textbf{FALSO} - De las tablas se concluye que el 71.4 \% de las razas de afectividad alta
se utilizan para compañia.
\item Hay catorce razas utilizadas para compañía.\\
\textbf{FALSO} - Hay 10 razas utilizadas para compañia.
\begin{Schunk}
\begin{Soutput}
com hun uti 
 10   9   8 
\end{Soutput}
\end{Schunk}
\item Hay catorce razas de afectividad alta.\\
\textbf{VERDADERO}
\begin{Schunk}
\begin{Sinput}
> table(DogBreeds$AFFE)
\end{Sinput}
\begin{Soutput}
hig low 
 14  13 
\end{Soutput}
\end{Schunk}
\item En las razas utilizadas para compañía hay catorce de afectividad alta.\\
\textbf{FALSO}
\item Todas las razas pesadas son de utilidad.\\
\textbf{VERDADERO} - Al observar la tabla se evicencia que Class/Cat=100.
\item Todas las razas de utilidad son de tamaño grande.\\
\textbf{VERDADERO} - Al observar la tabla se evicencia que Cat/Class=100.
\item La velocidad no caracteriza a las razas de caza.\\
\textbf{VERDADERO} - 
\item Hay cinco razas de utilidad.\\
\textbf{FALSO} - Hay 8 razas de utilidad.
\begin{Schunk}
\begin{Soutput}
com hun uti 
 10   9   8 
\end{Soutput}
\end{Schunk}
\end{enumerate}

\end{document}
