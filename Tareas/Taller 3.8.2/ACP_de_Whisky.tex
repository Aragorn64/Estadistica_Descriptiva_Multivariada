\documentclass[letterpaper, 12pt]{article}

\usepackage[utf8]{inputenc}
\usepackage{comment}
\usepackage{amsmath}  % Matrices
\usepackage{amsfonts} % Matrices
\usepackage{booktabs} % Tablas
\usepackage{paralist} % Enumeracion Horizontal
\usepackage{enumitem} % Listas

\title{ACP de ``Whisky”}
\author{Sebastian Aristizabal, Esteban Avendaño Forero,\\
Juan David Sarmiento.}
\date{\today}

\linespread{1.5}


\usepackage{Sweave}
\begin{document}
\Sconcordance{concordance:ACP_de_Whisky.tex:ACP_de_Whisky.Rnw:%
1 16 1 1 4 1 1 1 0 88 1}


\maketitle

\begin{center}
\textbf{Objetivo}
\end{center}

El objetivo es estudiar la relación calidad-precio de 35 marcas de whisky
utilizando las variables precio (francos franceses), proporción de malta (%),
vejez (añejamiento en años) y apreciación (nota promedio de un panel de
catadores redondeada a entera). Se dispone además de una variable nominal
categoría, que clasifica las marcas según su contenido de malta (1 = Bajo, 2
= Estándar, 3 = Puro malta) (Fine, 1996).

\newpage
\begin{center}
\textbf{Preguntas}
\end{center}

Realice primero un ACP no normado y luego un ACP normado y responda a las 
preguntas.

\begin{enumerate}

        \item En el ACP no normado, analice la contribución de las variables 
        a la inercia. ¿Realmente se puede considerar un análisis de las cuatro 
        variables?
        
        \item Analice la matriz de varianzas y covarianzas con la ayuda del 
        primer plano factorial de las variables. Haga un resumen (interpretación 
        del primer plano factorial de las variables).
        
        \item Realice el ACP normado y justifique por qué es el que conviene 
        para los objetivos de este taller.
        
        \item ¿Cuántos ejes retiene para el análisis? ¿Por qué?
        
        \item ¿Cuál es la variable que más contribuye al primer eje? ¿Cuál 
        es la que menos? (indique los porcentajes).
        
        \item Según el círculo de correlaciones, ¿cuáles son las variables 
        más correlacionadas? ¿Cuánto es el valor de la correlación? ¿Sí 
        corresponden a lo que se observa en la matriz de correlaciones?
        
        \item ¿Cuál es la variable mejor representada en el primer plano 
        factorial? ¿Cuál la peor? (Escriba los porcentajes).
        
        \item ¿Qué representa el primer eje? ¿Qué nombre le asignaría? ¿Qué 
        representa el segundo eje?
        
        \item ¿Cuál es el individuo mejor representado en el primer plano 
        factorial? Ubique sobre el gráfico de individuos al peor representado 
        sobre el primer plano factorial (indique los porcentajes).
        
        \item Supongamos que usted tiene una gráfica de individuos, donde 
        no semuestran los antiguos ejes de las variables. ¿Cómo dibuja los ejes 
        de apreciación y de precio? (Responda concretamente, es decir, con 
        números).
        
        \item ¿Qué características tienen las marcas de whisky según sus 
        ubicaciones en el plano (a la derecha, a la izquierda, arriba, abajo)?
        
        \item ¿Qué significa el círculo del primer plano factorial de 
        variables? ¿Cómo lo dibujaría en una gráfica impresa donde no está? 
        (Suponga quelas escalas de los dos ejes son iguales).
        
        \item A partir de la posición en el plano deduzca las características 
        de las tres categorías de whisky (bajo, estándar y pura malta).
        
        \item Supongamos que usted desea comprar una botella de whisky con 
        buena apreciación y que no sea tan cara. Dé dos números de marcas que
        compraría. ¿Por qué? ¿Cuáles son las características de las dos marcas?
        
        \item Seleccione dos marcas que definitivamente no compraría. 
        ¿Por qué? ¿Qué características tienen?
        
        \item Realice un resumen práctico del análisis suponiendo que lo va
        a entregar a una compañía que contrató el estudio. Debe dar respuesta al
        objetivo y apoyarse en las tablas y gráficas que crea necesarias.
        

\end{enumerate}




\end{document}
