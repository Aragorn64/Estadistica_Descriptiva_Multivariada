\documentclass[letterpaper, 12pt]{article}
\usepackage[utf8]{inputenc}
\usepackage{comment}
\usepackage{amsmath}  % Matrices
\usepackage{amsfonts} % Matrices
\usepackage{booktabs} % Tablas
                                                                                
\title{Tarea \#2}
\author{Esteban Avendaño Forero}
\date{30 de agosto}

                                
\linespread{1.5}
\usepackage{Sweave}
\begin{document}
\Sconcordance{concordance:Tarea.tex:Tarea.Rnw:%
1 11 1 1 0 14 1}


\maketitle

\begin{enumerate}
\setcounter{enumi}{1}
\item Instale R y a partir del manual de introducción conteste:
\begin{enumerate}
\item[2.1] ¿En R hay diferencia entre mayúsculas y minúsculas?\\
Si, R si diferencia entre las mayusculas y las minusculas. Por ejemplo, la 
variable a es diferente a la variable A.
\item[2.2] ¿Con qué se separan las instrucciones de R?\\
Si uno va a relizar varias instrucciones seguidas en R, se debe separar cada 
instruccion con un punto y coma (;) entre ellas.
\item[2.3] ¿Cómo se escriben comentarios en R?\\
Los comentarios en R se pueden escribir con el simbolo \# antes del comentario.
\item[2.4] ¿Qué significa cuando aparece + luego de teclear Enter?\\
Cuando aparece el simbolo + luego de teclear enter, significa que el programa 
esta esperando algun otro comando para terminar de ejecutar la orden dada.
\item[2.5] ¿Cómo se recuerdan comandos tecleados previamente en R?\\
Para recordar los comandos previamente utilizados en R, se debe utilizar la 
flecha hacia arriba y automaticamente aparecerean los comandos utilizados 
anteriormente.
\item[2.6] ¿Qué es el workspace?\\
El workspace es el entorno de trabajo donde se esta utilizando en programa R; 
aqui se almacenan los objetos definidos por el usuario (vectores, matrices, 
entre otros).
\item[2.7] ¿Qué se almacena en .RData?, ¿qué en .Rhistory?\\
En los archivos .RData se pueden almacenar los objetos que el usuario decida 
tales como dataframes, matrices, vectores, entre otros. En el caso de .Rhistory,
este archivo almacena el historial de comandos ejecutados por el usuario en una 
sesion.
\item[2.8] ¿Cómo se obtiene ayuda en R para una función específica?\\
Para obtener ayuda para una funcion especifica en R, se puede utilizar el signo
de interrogacion previo a la funcion (?mean()), o con ayuda de help().
\item[2.9] ¿Cuáles son los símbolos de comparación en R: menor que, menor o 
igual, mayor, mayor o igual, igual y diferente?\\
Menor: <, Menor o Igual: <=, Mayor: >, Mayor o Igual: >=, Diferente: != . 
\item[2.10] ¿Cuáles son los operadores lógicos: OR , AND y negación?\\
OR: |, AND: \&, Negacion: !.
\item[2.11] ¿Qué efecto tienen  al imprimir una cadena de caracteres?
\item[2.12] ¿Cuáles son los principales objetos de R?\\
Los principales objetos en R son vectores, matrices, listas, funciones, 
dataframes.
\item[2.13] ¿Cómo se define un escalar en R?\\
Para definir un escalar en R, hay que definirlo y declararlo dentro de una 
variable. Por ejemplo, para definir el escalar 5 con la letra a, debemos 
definirlo con el signo igual (a=5) o con una flecha (a <- 5).
\item[2.14] ¿Qué es un factor y qué atributos tiene?\\
Un factor es un objeto de R que se utiliza para categorizar los datos y 
almacenarlos en diferentes niveles. Se pueden almacenar tanto caracteres como 
numeros.
\item[2.15] ¿Qué hace la función tapply ?\\
La funcion tapply permite aplicar una funcion a los componentes de un vector, 
utilizando un parametro determinado. 
\end{enumerate}

\item Escriba para cada instrucción un comentario resumiendo lo que hace
cada función:
\begin{enumerate}
\item[3.1] help.start():\\
Abre el manual de ayuda de R, donde se pueden consultar los manuales, referencias 
y material adicional.
\item[3.2] sink("record.lis")\\
La funcion sink("record.lis") guardara la salida del programa en el archivo 
llamado record.lis.
\item[3.3] misdatos <-read.table('data.dat'):\\ 
Lee los datos de la tabla data.dat y los almacena en una variable llamada mis 
datos.
\item[3.4] L2 <- list(A=x, B=y):\\
Crea una variable L2, la cual es una lista con los elementos A y B.
\item[3.5] ts(1:47, frequency = 12, start = c(1959, 2))
\item[3.6] exp1 <- expression(x /(y + exp(z)))
\item[3.7] x <- rpois(40, lambda=5):\\
Crea un vector x cuyas componentes son 40 numeros generados aleatoriamente 
provenientes de una distribucion Poission con parametro $\lambda$=5.
\item[3.8] x[x \%\%2 == 0]\\
Aqui se comprueba cuales numeros del vector x modulo 2 dan residuo de 0, y se 
devuelven aquellos numeros que cumplen con esa condicion.
\item[3.9] x <- rnorm(50)\\
Crea un vector x cuyas componentes son 50 numeros generados aleatoriamente 
provenientes de una distribucion Normal con media 0 y desviacion estandar 1.
\item[3.10] mean(x)\\
Aqui se obtiene la media del vector x.
\end{enumerate}

\item Suponga que usted es la consola de R. Responda al frente a cada uno
de los comandos:
\begin{enumerate}
\item[4.1] 0/0

\begin{Schunk}
\begin{Soutput}
[1] NaN
\end{Soutput}
\end{Schunk}

\item[4.2] labs <- paste(c('X','Y'), 1:10, sep='');labs

\begin{Schunk}
\begin{Soutput}
 [1] "X1"  "Y2"  "X3"  "Y4"  "X5"  "Y6"  "X7"  "Y8" 
 [9] "X9"  "Y10"
\end{Soutput}
\end{Schunk}

\item[4.3] c("x","y")[rep(c(1,2,2,1), times=4)]

\begin{Schunk}
\begin{Soutput}
 [1] "x" "y" "y" "x" "x" "y" "y" "x" "x" "y" "y" "x"
[13] "x" "y" "y" "x"
\end{Soutput}
\end{Schunk}

\item[4.4] ls()

\begin{Schunk}
\begin{Soutput}
[1] "labs"
\end{Soutput}
\end{Schunk}

\item[4.5] apropos("eigen")

\begin{Schunk}
\begin{Soutput}
[1] "eigen"       "print.eigen"
\end{Soutput}
\end{Schunk}

\item[4.6] x <- 1; mode(x)
\begin{Schunk}
\begin{Soutput}
[1] "numeric"
\end{Soutput}
\end{Schunk}

\item[4.7] seq(1, 5, 0.5)
\begin{Schunk}
\begin{Soutput}
[1] 1.0 1.5 2.0 2.5 3.0 3.5 4.0 4.5 5.0
\end{Soutput}
\end{Schunk}

\item[4.8] gl(3, 5)
\begin{Schunk}
\begin{Soutput}
 [1] 1 1 1 1 1 2 2 2 2 2 3 3 3 3 3
Levels: 1 2 3
\end{Soutput}
\end{Schunk}

\item[4.9] expand.grid(a=c(60,80), p=c(100, 300),\\
sexo=c("Macho", "Hembra"))-> trat\\
trat;dim(trat);class(trat)

\begin{Schunk}
\begin{Soutput}
   a   p   sexo
1 60 100  Macho
2 80 100  Macho
3 60 300  Macho
4 80 300  Macho
5 60 100 Hembra
6 80 100 Hembra
7 60 300 Hembra
8 80 300 Hembra
\end{Soutput}
\begin{Soutput}
[1] 8 3
\end{Soutput}
\begin{Soutput}
[1] "data.frame"
\end{Soutput}
\end{Schunk}

\item[4.10] v <- c(10, 20, 30);diag(v)
\begin{Schunk}
\begin{Soutput}
     [,1] [,2] [,3]
[1,]   10    0    0
[2,]    0   20    0
[3,]    0    0   30
\end{Soutput}
\end{Schunk}

\end{enumerate}
\end{enumerate}


\end{document}

