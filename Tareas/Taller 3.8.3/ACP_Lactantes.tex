\documentclass[letterpaper, 12pt]{article}

\usepackage[utf8]{inputenc}
\usepackage{comment}
\usepackage{amsmath}  % Matrices
\usepackage{amsfonts} % Matrices
\usepackage{booktabs} % Tablas
\usepackage{paralist} % Enumeracion Horizontal
\usepackage{enumitem} % Listas

\title{ACP ``Lactantes”}
\author{Sebastian Aristizabal, Esteban Avendaño Forero,\\
Juan David Sarmiento.}
\date{\today}

\linespread{1.5}


\usepackage{Sweave}
\begin{document}
\Sconcordance{concordance:ACP_Lactantes.tex:ACP_Lactantes.Rnw:%
1 16 1 1 4 1 1 1 0 110 1}


\maketitle

De Dalgaard (2008) se tomó el ejemplo kfm-Breast-feeding data, cuyos datos
están en el objeto kfm{ISwR} y son una tabla de 50 filas (bebés de aproxima-
damente 2 meses) y 6 columnas (Dalgaard, 2020).\\

Las variables continuas son: leche = leche materna consumida por el
niño: dl/24 horas; peso = peso del niño, kg; tetero = alimentación suple-
mentaria, ml/24 horas; peso.madre, kg; talla.madre, cm. Se dispone de la
variable categórica sexo (masculino, femenino). Se plantea realizar un ACP
que responda a los objetivos siguientes:
\begin{enumerate}

        \item Descripción de los bebés según su peso, consumo de leche materna
        y tetero, y su relación con el peso y talla de las madres.
        \item ¿Está relacionada la alimentación suplementaria (tetero) con las 
        demás variables?
        \item ¿Hay diferencias entre niños y niñas?
        
\end{enumerate}

Conteste a las siguientes preguntas:

\newpage
\begin{center}
\textbf{Preguntas}
\end{center}

Realice primero un ACP no normado y luego un ACP normado y responda a las 
preguntas.

\begin{enumerate}

        \item ¿Por qué con el ACP se cumplen los objetivos planteados?
        
        \item ¿Realiza un ACP normado o no normado? ¿Por qué?
        
        \item Describa el bebé promedio según las cinco variables.
        
        \item ¿Cuántos ejes retiene para el análisis? ¿Por qué?
        
        \item ¿Qué variables se puede decir que están más altamente correlaciona
        das con el primer factor? ¿Puede darle algún significado a este primer
        factor?
        
        \item ¿Puede identificar subconjuntos de variables altamente 
        correlacionas entre sí? ¿Existe algún subconjunto de variables que se
        pueda decir que no está correlacionado con otro subconjunto de variables?
        
        \item ¿Qué características tienen los lactantes según su posición en el 
        primer plano factorial?
        
        \item ¿Los análisis anteriores sugieren que pueden constituirse grupos 
        de bebés? ¿Podría sugerir algunos?
        
        \item ¿Se puede decir que hay diferencia entre niños y niñas en este 
        análisis? ¿Cuáles son son esas diferencias?
        
        \item Escriba un resumen práctico del análisis que satisfaga los
        objetivos planteados.
        
\end{enumerate}

Las preguntas que siguen son sobre lectura y algunos cálculos en el ejemplo 
``Lactantes”. Responda a los ¿por qué? mencionando la manera cómo dedujo la 
respuesta: ayuda que utilizó, la gráfica que leyó, etc.

\begin{enumerate}[resume]
        
        \item La inercia de las nubes de puntos asociadas al ACP es:
        
        \item Primer valor propio:
        
        \item Primer vector propio:
        
        \item Asdfdsgffdg
        
        \item Correlación entre tetero y primer factor:
        
        \item Variable que más contribuye al primer eje: ¿Porque?
        
        \item ¿Las dos variables menos correlacionadas con tetero son: ¿Porque?
        
        \item Variable mejor representada en el primer plano factorial: ¿Porque?
        
        \item Coordenadas del bebé promedio sobre el primer plano factorial:
        
        \item Los dos bebés que más tetero consumen son:
        
        \item Para el bebé situado en el extremo superior del primer plano 
        factorial escriba las coordenadas sobre los dos primeros ejes factoriales:
        
        \item Calcule la contribución del bebé anterior a la inercia del segundo
        eje factorialy la calidad de representación sobre el mismo eje.
         
        \item Escriba las coordenadas de los antiguos ejes unitarios de las 
        variables \emph{leche} y \emph{tetero} sobre el primer plano factorial.
        
        \item Dibuje los antiguos ejes de \emph{leche} y \emph{tetero} sobre el primer plano 
        factorial, indicando los lados positivos y negativos.
        

\end{enumerate}




\end{document}
