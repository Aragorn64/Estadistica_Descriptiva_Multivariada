\documentclass[letterpaper, 12pt]{article}

\usepackage[utf8]{inputenc}
\usepackage{comment}
\usepackage{amsmath}  % Matrices
\usepackage{amsfonts} % Matrices
\usepackage{booktabs} % Tablas
\usepackage{paralist} % Enumeracion Horizontal
\usepackage{enumitem} % Listas

\title{Análisis en coordenadas principales}
\author{Sebastian Aristizabal, Esteban Avendaño Forero,\\
Juan David Sarmiento.}
\date{\today}

\linespread{1.5}


\usepackage{Sweave}
\begin{document}
\Sconcordance{concordance:Analisis_Coordenadas_Principales.tex:Analisis_Coordenadas_Principales.Rnw:%
1 16 1 1 4 1 1 1 0 44 1}


\maketitle

En Hidalgo et al. (2007) se construye una distancia cultural entre algunos
países latinoaméricanos. En la tabla 4.3 se presentan las distancias, que co-
rresponden a la raíz cuadrada de las presentadas en el artículo.\\

\newpage
Realice el análisis en coordenadas principales ( ACO ) sobre la matriz de
distancias, utilizando las funciones dudi.pco e inertia.dudi de ade4 y res-
ponda a las siguientes preguntas:

\begin{enumerate}

        \item ¿Cuál es la dimensión del espacio de representación?
        
        \item ¿Cuántos ejes selecciona para el análisis? ¿Por qué?
        
        \item ¿Se pueden obtener ayudas para la interpretación de las variables?
        
        \item ¿Algunos países tienen una calidad de representación en el primer
        plano factorial inferior al 10 \%? ¿Cuáles?
        
        \item ¿Qué países tienen una contribución al primer eje por encima del
        promedio?
        
        \item Analice los planos factoriales. Puede utilizar plot.dudi{FactoClass}
        para graficar los planos factoriales que requiera.
        
        \item A partir de los planos factoriales establezca una partición de 
        los países. Describa comparativamente los grupos de países formados.
        
        \item Compare los resultados con los del artículo.
        
        \item Haga un resumen práctico del análisis.
        

\end{enumerate}




\end{document}
