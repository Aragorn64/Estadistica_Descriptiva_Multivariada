\documentclass[letterpaper, 12pt]{article}

\usepackage[utf8]{inputenc}
\usepackage{comment}
\usepackage{amsmath}  % Matrices
\usepackage{amsfonts} % Matrices
\usepackage{booktabs} % Tablas
\usepackage{paralist} % Enumeracion Horizontal
\usepackage{enumitem} % Listas

\title{Imagen euclidiana de matrices de varianzas-covarianzas y de correlaciones}
\author{Sebastian Aristizabal, Esteban Avendaño Forero,\\
Juan David Sarmiento.}
\date{\today}

\linespread{1.5}


\usepackage{Sweave}
\begin{document}
\Sconcordance{concordance:Imagen_Euclidiana.tex:Imagen_Euclidiana.Rnw:%
1 16 1 1 4 1 1 1 0 43 1}


\maketitle

En el artículo de Correa, De Rosa \& Lesino (2006) se realiza un análisis del
clima en la ciudad de Mendoza (Argentina). El artículo no presenta los da-
tos originales pero tiene dos matrices de correlaciones. El ejercicio de este
taller consiste en construir los círculos de correlaciones a partir de las ma-
trices y comparar con los resultados del artículo. Para el caso de la matriz de
correlaciones que corresponde a un día de primavera por la mañana (tabla
4.2), responda las siguientes preguntas:

\newpage
\begin{center}
\textbf{Preguntas}
\end{center}

\begin{enumerate}

        \item Objetivo de análisis.
        
        \item Descripción de las variables.
        
        \item ¿Cuáles son las unidades estadísticas (“individuos”) en el análisis?
        
        \item ¿Se pueden obtener coordenadas y ayudas para la interpretación de
        los “individuos” cuando solo se tienen las matrices de correlaciones?
        ¿Por qué?
        
        \item ¿Cuántos ejes selecciona para analizar? ¿Por qué?
        
        \item ¿Qué significado le puede dar cada uno de los ejes que va a analizar?
        
        \item Grafique y analice los planos factoriales que estime conveniente.
        
        \item Resuma el análisis respondiendo a los objetivos.

\end{enumerate}




\end{document}
